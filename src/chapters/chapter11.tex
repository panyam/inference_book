%%%%%%%%%%%%%%%%%%%%% chapter11.tex %%%%%%%%%%%%%%%%%%%%%%%%%%%%%%%%%
%
% Chapter 11: Multi-Tenant Architecture
%
%%%%%%%%%%%%%%%%%%%%%%%% Springer-Verlag %%%%%%%%%%%%%%%%%%%%%%%%%%

\chapter{Multi-Tenant Architecture}
\label{ch:multi-tenant}

\abstract*{Serving multiple organizations requires careful architectural decisions. This chapter covers tenant isolation strategies, data separation, resource allocation, and the organizational model that underlies a multi-tenant inference platform. We design a system that provides strong isolation while efficiently sharing underlying infrastructure.}

\abstract{Serving multiple organizations requires careful architectural decisions. This chapter covers tenant isolation strategies, data separation, resource allocation, and the organizational model that underlies a multi-tenant inference platform. We design a system that provides strong isolation while efficiently sharing underlying infrastructure.}

% =============================================================================
\section{Multi-Tenancy Concepts}
\label{sec:mt-concepts}
% =============================================================================

% TODO: Foundational concepts
% - What is multi-tenancy
% - Why it matters for inference
% - Isolation vs sharing trade-offs

% =============================================================================
\section{Isolation Models}
\label{sec:isolation-models}
% =============================================================================

\subsection{Shared Everything}
\label{subsec:shared-everything}

% TODO: Single pool approach
% - All tenants share resources
% - Simple but risky
% - Noisy neighbor problem

\subsection{Namespace Isolation}
\label{subsec:namespace-isolation}

% TODO: Logical separation
% - Separate data but shared compute
% - Good balance for most cases

\subsection{Dedicated Resources}
\label{subsec:dedicated-resources}

% TODO: Physical separation
% - Dedicated GPUs per tenant
% - Maximum isolation
% - Enterprise tier

% =============================================================================
\section{Tenant Data Model}
\label{sec:tenant-model}
% =============================================================================

\begin{programcode}{Tenant Data Model}
\begin{lstlisting}[language=Go]
// internal/tenant/models.go

type Tenant struct {
    ID            string    `json:"id"`
    Name          string    `json:"name"`
    Tier          TenantTier `json:"tier"`
    Status        TenantStatus `json:"status"`
    Settings      TenantSettings `json:"settings"`
    ResourceQuota ResourceQuota `json:"resource_quota"`
    CreatedAt     time.Time `json:"created_at"`
}

type TenantTier string

const (
    TierFree       TenantTier = "free"
    TierStarter    TenantTier = "starter"
    TierPro        TenantTier = "pro"
    TierEnterprise TenantTier = "enterprise"
)

type TenantSettings struct {
    AllowedModels     []string `json:"allowed_models"`
    MaxContextLength  int      `json:"max_context_length"`
    CustomEndpoint    bool     `json:"custom_endpoint"`
    DataRetention     int      `json:"data_retention_days"`
    IsolationLevel    string   `json:"isolation_level"`
}

type ResourceQuota struct {
    RequestsPerMinute   int   `json:"requests_per_minute"`
    TokensPerMonth      int64 `json:"tokens_per_month"`
    MaxConcurrent       int   `json:"max_concurrent"`
    DedicatedGPUs       int   `json:"dedicated_gpus"`
}
\end{lstlisting}
\end{programcode}

% =============================================================================
\section{Request Context}
\label{sec:request-context}
% =============================================================================

% TODO: Propagating tenant info
% - Tenant in context
% - Throughout request lifecycle

% =============================================================================
\section{Resource Allocation}
\label{sec:resource-allocation}
% =============================================================================

% TODO: Fair sharing

\subsection{Shared Pool Model}
\label{subsec:shared-pool}

% TODO: Default approach

\subsection{Reserved Capacity}
\label{subsec:reserved-capacity}

% TODO: Guaranteed minimums

\subsection{Dynamic Allocation}
\label{subsec:dynamic-allocation}

% TODO: Based on demand

% =============================================================================
\section{Noisy Neighbor Prevention}
\label{sec:noisy-neighbor}
% =============================================================================

% TODO: Preventing one tenant from affecting others
% - Rate limiting per tenant
% - Resource limits
% - Queue isolation

% =============================================================================
\section{Summary}
\label{sec:ch11-summary}
% =============================================================================

\begin{important}{Key Takeaways}
\begin{itemize}
\item Multi-tenancy enables efficient resource sharing
\item Isolation level should match tenant tier
\item Noisy neighbor problems require proactive prevention
\item Tenant context must flow through the entire request
\end{itemize}
\end{important}

\section*{Problems}
\addcontentsline{toc}{section}{Problems}

\begin{prob}
\label{prob:ch11-tenant-service}
Implement a \texttt{TenantService} that manages tenant lifecycle including creation, updates, and suspension.
\end{prob}

\input{chapters/references11}
