%%%%%%%%%%%%%%%%%%%%% chapter07.tex %%%%%%%%%%%%%%%%%%%%%%%%%%%%%%%%%
%
% Chapter 7: Rate Limiting and Quotas
%
%%%%%%%%%%%%%%%%%%%%%%%% Springer-Verlag %%%%%%%%%%%%%%%%%%%%%%%%%%

\chapter{Rate Limiting and Quotas}
\label{ch:rate-limiting}

\abstract*{Protecting your inference infrastructure requires thoughtful rate limiting. This chapter covers token bucket and sliding window algorithms, per-user and per-tier quotas, burst handling, and graceful degradation. We implement a flexible rate limiting system that integrates with authentication and provides clear feedback to users.}

\abstract{Protecting your inference infrastructure requires thoughtful rate limiting. This chapter covers token bucket and sliding window algorithms, per-user and per-tier quotas, burst handling, and graceful degradation. We implement a flexible rate limiting system that integrates with authentication and provides clear feedback to users.}

% =============================================================================
\section{Why Rate Limiting?}
\label{sec:why-rate-limiting}
% =============================================================================

% TODO: Motivation
% - Protect infrastructure
% - Fair resource sharing
% - Cost management
% - Prevent abuse
% - SLA enforcement

% =============================================================================
\section{Rate Limiting Algorithms}
\label{sec:rl-algorithms}
% =============================================================================

\subsection{Token Bucket}
\label{subsec:token-bucket}

% TODO: Token bucket explained
% - How it works
% - Allows bursting
% - Implementation

\subsection{Sliding Window}
\label{subsec:sliding-window}

% TODO: Sliding window
% - Fixed vs sliding
% - Memory considerations
% - Implementation

\subsection{Leaky Bucket}
\label{subsec:leaky-bucket}

% TODO: Leaky bucket
% - Constant rate output
% - When to use

\subsection{Algorithm Comparison}
\label{subsec:algorithm-comparison}

% TODO: Comparison table

% =============================================================================
\section{Designing Rate Limit Policies}
\label{sec:rl-policies}
% =============================================================================

% TODO: Policy design

\subsection{Per-User Limits}
\label{subsec:per-user-limits}

% TODO: User-based limits
% - Requests per minute
% - Tokens per minute
% - Concurrent requests

\subsection{Per-Tier Limits}
\label{subsec:per-tier-limits}

% TODO: Tier-based differentiation
% - Free tier
% - Pro tier
% - Enterprise tier

\begin{programcode}{Rate Limit Configuration}
\begin{lstlisting}[language=Go]
// internal/ratelimit/config.go

type RateLimitConfig struct {
    Tiers map[string]TierLimits `yaml:"tiers"`
}

type TierLimits struct {
    RequestsPerMinute    int `yaml:"requests_per_minute"`
    TokensPerMinute      int `yaml:"tokens_per_minute"`
    TokensPerDay         int `yaml:"tokens_per_day"`
    MaxConcurrentReqs    int `yaml:"max_concurrent_requests"`
    MaxTokensPerRequest  int `yaml:"max_tokens_per_request"`
    BurstMultiplier      float64 `yaml:"burst_multiplier"`
}

var DefaultTierLimits = map[string]TierLimits{
    "free": {
        RequestsPerMinute:    10,
        TokensPerMinute:      1000,
        TokensPerDay:         10000,
        MaxConcurrentReqs:    2,
        MaxTokensPerRequest:  500,
        BurstMultiplier:      1.5,
    },
    "pro": {
        RequestsPerMinute:    60,
        TokensPerMinute:      10000,
        TokensPerDay:         500000,
        MaxConcurrentReqs:    10,
        MaxTokensPerRequest:  4000,
        BurstMultiplier:      2.0,
    },
    "enterprise": {
        RequestsPerMinute:    1000,
        TokensPerMinute:      100000,
        TokensPerDay:         10000000,
        MaxConcurrentReqs:    100,
        MaxTokensPerRequest:  32000,
        BurstMultiplier:      3.0,
    },
}
\end{lstlisting}
\end{programcode}

% =============================================================================
\section{Implementing Rate Limiting}
\label{sec:rl-implementation}
% =============================================================================

% TODO: Go implementation

\begin{programcode}{Rate Limiter Interface}
\begin{lstlisting}[language=Go]
// internal/ratelimit/limiter.go

package ratelimit

import (
    "context"
    "time"
)

type Decision struct {
    Allowed       bool
    Remaining     int
    ResetAt       time.Time
    RetryAfter    time.Duration
    LimitedBy     string // "rpm", "tpm", "concurrent", etc.
}

type Limiter interface {
    // Allow checks if a request should be allowed
    Allow(ctx context.Context, userID string, tier string) Decision

    // AllowN checks if N tokens can be consumed
    AllowN(ctx context.Context, userID string, tier string, tokens int) Decision

    // Release releases a concurrent request slot
    Release(ctx context.Context, userID string)
}
\end{lstlisting}
\end{programcode}

\subsection{Redis-based Implementation}
\label{subsec:redis-implementation}

% TODO: Distributed rate limiting
% - Why Redis
% - Lua scripts for atomicity
% - Implementation

\subsection{In-Memory Implementation}
\label{subsec:inmemory-implementation}

% TODO: Single-node option
% - When appropriate
% - sync.Map approach

% =============================================================================
\section{Rate Limiting Middleware}
\label{sec:rl-middleware}
% =============================================================================

% TODO: HTTP middleware

\begin{programcode}{Rate Limit Middleware}
\begin{lstlisting}[language=Go]
// internal/ratelimit/middleware.go

func RateLimitMiddleware(limiter Limiter) func(http.Handler) http.Handler {
    return func(next http.Handler) http.Handler {
        return http.HandlerFunc(func(w http.ResponseWriter, r *http.Request) {
            user := auth.GetUser(r.Context())
            if user == nil {
                http.Error(w, "unauthorized", http.StatusUnauthorized)
                return
            }

            decision := limiter.Allow(r.Context(), user.ID, user.Tier)

            // Set rate limit headers
            w.Header().Set("X-RateLimit-Limit", fmt.Sprintf("%d", decision.Limit))
            w.Header().Set("X-RateLimit-Remaining", fmt.Sprintf("%d", decision.Remaining))
            w.Header().Set("X-RateLimit-Reset", fmt.Sprintf("%d", decision.ResetAt.Unix()))

            if !decision.Allowed {
                w.Header().Set("Retry-After", fmt.Sprintf("%d", int(decision.RetryAfter.Seconds())))
                http.Error(w, "rate limit exceeded", http.StatusTooManyRequests)
                return
            }

            next.ServeHTTP(w, r)
        })
    }
}
\end{lstlisting}
\end{programcode}

% =============================================================================
\section{Token-Based Rate Limiting}
\label{sec:token-rate-limiting}
% =============================================================================

% TODO: Limiting by tokens, not just requests
% - Why tokens matter for LLMs
% - Pre-request estimation
% - Post-request adjustment

% =============================================================================
\section{Handling Bursts}
\label{sec:handling-bursts}
% =============================================================================

% TODO: Burst strategies
% - Allow short bursts
% - Queuing excess requests
% - Graceful degradation

% =============================================================================
\section{Quota Management}
\label{sec:quota-management}
% =============================================================================

% TODO: Daily/monthly quotas
% - Tracking usage
% - Alerting on approach
% - Hard vs soft limits

% =============================================================================
\section{User Feedback}
\label{sec:rl-user-feedback}
% =============================================================================

% TODO: Clear communication
% - Rate limit headers
% - Error messages
% - Dashboard visibility

% =============================================================================
\section{Summary}
\label{sec:ch07-summary}
% =============================================================================

\begin{important}{Key Takeaways}
\begin{itemize}
\item Rate limiting protects infrastructure and ensures fair usage
\item Token bucket allows bursting while maintaining average rate
\item Token-based limits matter more than request counts for LLMs
\item Clear headers and error messages improve user experience
\item Redis enables distributed rate limiting across multiple nodes
\end{itemize}
\end{important}

% =============================================================================
\section*{Problems}
\addcontentsline{toc}{section}{Problems}
% =============================================================================

\begin{prob}
\label{prob:ch07-token-bucket}
\textbf{Token Bucket Implementation}\\
Implement a token bucket rate limiter from scratch. Support configurable fill rate, bucket size, and burst capacity.
\end{prob}

\begin{prob}
\label{prob:ch07-redis-limiter}
\textbf{Redis Rate Limiter}\\
Implement a distributed rate limiter using Redis. Use Lua scripts to ensure atomic check-and-decrement operations.
\end{prob}

\begin{prob}
\label{prob:ch07-quota-tracker}
\textbf{Quota Tracking System}\\
Build a system that tracks daily and monthly token quotas per user. Include alerts when users approach their limits.
\end{prob}

%%%%%%%%%%%%%%%%%%%%%%%% references02.tex %%%%%%%%%%%%%%%%%%%%%%%%%%%%%%
% References for Chapter 2
%%%%%%%%%%%%%%%%%%%%%%%% Springer-Verlag %%%%%%%%%%%%%%%%%%%%%%%%%%

\begin{thebibliography}{99.}

\bibitem{nvidia2024h100} NVIDIA: H100 Tensor Core GPU Datasheet. \url{https://www.nvidia.com/en-us/data-center/h100/} (2024)

\bibitem{nvidia2024a100} NVIDIA: A100 Tensor Core GPU Datasheet. \url{https://www.nvidia.com/en-us/data-center/a100/} (2024)

\bibitem{amd2024mi300} AMD: Instinct MI300X Accelerator. \url{https://www.amd.com/en/products/accelerators/instinct/mi300/mi300x.html} (2024)

\bibitem{apple2024m4} Apple: M4 Chip Architecture. \url{https://www.apple.com/newsroom/2024/05/apple-introduces-m4-chip/} (2024)

\bibitem{runpod2025pricing} RunPod: GPU Cloud Pricing. \url{https://www.runpod.io/gpu-instance/pricing} (accessed December 2025)

\bibitem{vastai2025} Vast.ai: GPU Marketplace. \url{https://vast.ai/} (accessed December 2025)

\bibitem{lambdalabs2025} Lambda Labs: GPU Cloud. \url{https://lambdalabs.com/service/gpu-cloud} (accessed December 2025)

\end{thebibliography}

