%%%%%%%%%%%%%%%%%%%%% chapter06.tex %%%%%%%%%%%%%%%%%%%%%%%%%%%%%%%%%
%
% Chapter 6: Building Control Plane v0.1
%
%%%%%%%%%%%%%%%%%%%%%%%% Springer-Verlag %%%%%%%%%%%%%%%%%%%%%%%%%%

\chapter{Building Control Plane v0.1}
\label{ch:control-plane-v01}

\abstract*{This chapter is where theory meets practice. We design and build the first version of our control plane---a foundation that will evolve throughout the book. Rather than a language tutorial, we focus on architecture, abstractions, and the reasoning behind design decisions. The complete implementation is available at \texttt{github.com/inference-book/inference-plane}, and by the end of this chapter you'll have a working inference system with observability built in from day one.}

\abstract{This chapter is where theory meets practice. We design and build the first version of our control plane---a foundation that will evolve throughout the book. Rather than a language tutorial, we focus on architecture, abstractions, and the reasoning behind design decisions. The complete implementation is available at \texttt{github.com/inference-book/inference-plane}, and by the end of this chapter you'll have a working inference system with observability built in from day one.}

% =============================================================================
\section{Setting Up Your Lab Environment}
\label{sec:lab-environment}
% =============================================================================

Before we start building, let's address a practical question: where should you run these examples?

\subsection{Local vs. Cloud: Our Recommendation}
\label{subsec:local-vs-cloud}

The previous chapters could be followed on any machine---reading model files, understanding architectures, and exploring inference engines requires minimal resources. But from this chapter forward, we'll be running actual inference workloads, and that changes things.

\textbf{We recommend using cloud GPU providers} for the hands-on portions of this book. Here's why:

\begin{description}
\item[Consistency] Every reader gets the same environment. No ``works on my machine'' problems. When we say ``pull llama3.2:7b and run inference,'' you'll see the same performance whether you're in Tokyo or Toronto.

\item[Accessibility] Not everyone has access to a modern GPU. A MacBook with M1 can run small models, but a \$500/month cloud instance with an A100 runs circles around it---and you only pay for hours used.

\item[Production Realism] Self-hosting inference typically means cloud deployment. Learning to provision, configure, and manage cloud GPU instances is part of the skill set you're developing.

\item[Cost Efficiency] A 7B model runs well on a \$0.50/hour instance. Running all Part I exercises might cost \$5-10 total. Compare that to buying and maintaining GPU hardware.
\end{description}

\subsection{What You'll Need}
\label{subsec:what-youll-need}

For Part I of this book (Chapters 6-7):
\begin{itemize}
\item A cloud GPU instance with 16-24GB VRAM (A10, RTX 4090, or similar)
\item Ubuntu 22.04 or later (the default on most providers)
\item Docker and Docker Compose installed
\item SSH access to the instance
\end{itemize}

Appendix~\ref{ch:appendix-gpu-providers} provides step-by-step setup guides for five popular GPU providers: AWS, GCP, Lambda Labs, RunPod, and Vast.ai. We recommend Lambda Labs or RunPod for beginners due to their simple interfaces and competitive pricing.

\begin{svgraybox}
\textbf{Running Locally (If You Prefer)}

If you have suitable hardware (Apple Silicon Mac, NVIDIA GPU with 8GB+ VRAM, or 32GB+ RAM for CPU inference), the examples work locally too. We note where cloud-specific steps apply. Just be aware that your results may differ from the examples depending on your hardware.
\end{svgraybox}

% =============================================================================
\section{Why a Control Plane?}
\label{sec:why-control-plane}
% =============================================================================

You've learned how LLMs work, selected hardware, chosen models, and compared inference engines. Now it's time to build. But why not just call Ollama or vLLM directly from your application?

\subsection{The Direct Integration Problem}
\label{subsec:direct-integration}

Consider a typical application that needs LLM capabilities:

\begin{lstlisting}[language=bash]
# Your application calls Ollama directly
curl http://localhost:11434/api/generate \
  -d '{"model": "llama3.2:7b", "prompt": "Hello"}'
\end{lstlisting}

This works for prototyping, but problems emerge quickly:

\begin{description}
\item[Engine Lock-in] Your application code contains Ollama-specific URLs and request formats. Switching to vLLM means changing every call site.
\item[No Visibility] How many requests per second? What's the p99 latency? Which models are popular? You have no idea.
\item[No Control] Every request gets equal treatment. No rate limiting, no authentication, no prioritization.
\item[Fragile Operations] The engine goes down and your application crashes. No health checks, no graceful degradation.
\end{description}

\subsection{The Control Plane Solution}
\label{subsec:control-plane-solution}

In networking, a \emph{control plane} handles configuration and policy decisions while the \emph{data plane} moves actual traffic. We borrow this concept: our control plane manages \emph{how} requests are handled while inference engines handle computation.

\begin{description}
\item[Backend Abstraction] Your application talks to the control plane using a standard API (OpenAI-compatible). The control plane translates to whatever backend you're running.
\item[Centralized Policy] Authentication, rate limiting, caching, and logging happen in one place. Change policy without touching application code.
\item[Observability] Every request is measured. Latency histograms, error rates, token throughput---all available via OpenTelemetry, exportable to Grafana, Prometheus, or any OTLP-compatible backend.
\item[Resilience] Health checks detect backend failures. Future versions will add failover, retries, and circuit breakers.
\end{description}

% Diagram: Inference Engine Pipeline
% Shows the flow from request to response through engine components

\begin{figure}[htbp]
\centering
\begin{tikzpicture}[
    node distance=1.2cm and 1.5cm,
    box/.style={rectangle, draw, rounded corners, minimum width=2.2cm, minimum height=1cm, align=center, font=\small},
    arrow/.style={->, thick, >=stealth},
    label/.style={font=\footnotesize\itshape, text=gray}
]

% Top row: Input processing
\node[box, fill=blue!10] (request) {Request\\Handler};
\node[box, fill=green!10, right=of request] (tokenizer) {Tokenizer};
\node[box, fill=orange!10, right=of tokenizer] (forward) {Forward\\Pass};

% Bottom row: Output generation
\node[box, fill=purple!10, below=of forward] (kvcache) {KV Cache\\Manager};
\node[box, fill=red!10, below=of tokenizer] (sampler) {Sampler};
\node[box, fill=blue!10, below=of request] (response) {Response\\Streamer};

% Top row arrows
\draw[arrow] (request) -- (tokenizer);
\draw[arrow] (tokenizer) -- (forward);

% Down and across
\draw[arrow] (forward) -- (kvcache);
\draw[arrow] (kvcache) -- (sampler);
\draw[arrow] (sampler) -- (response);

% Loop back for autoregressive generation (curved path on the right side)
\draw[arrow, dashed, rounded corners=8pt]
    (kvcache.east) -- ++(0.8,0) -- ++(0,1.2) -- ++(-0.8,0) -- (forward.east);
\node[label, anchor=west] at ($(kvcache.east) + (0.9, 0.6)$) {next token};

% Input/Output labels
\node[left=0.5cm of request, font=\small] (input) {``Hello''};
\node[left=0.5cm of response, font=\small] (output) {``Hi!''};
\draw[arrow] (input) -- (request);
\draw[arrow] (response) -- (output);

% Component labels
\node[above=0.15cm of request, label] {API/Queue};
\node[above=0.15cm of tokenizer, label] {Text $\rightarrow$ IDs};
\node[above=0.15cm of forward, label] {GPU Compute};
\node[below=0.15cm of kvcache, label] {Memory Mgmt};
\node[below=0.15cm of sampler, label] {Token Select};
\node[below=0.15cm of response, label] {SSE Stream};

\end{tikzpicture}
\caption{Inference engine pipeline. Requests flow through tokenization and the forward pass (top row), then through KV cache management and sampling (bottom row). The dashed arrow shows the autoregressive loop---each generated token feeds back through the forward pass until generation completes.}
\label{fig:inference-pipeline}
\end{figure}


% =============================================================================
\section{v0.1 Scope and Architecture}
\label{sec:v01-architecture}
% =============================================================================

Control Plane v0.1 is intentionally minimal---enough to be useful while establishing patterns for growth.

\subsection{What We're Building}
\label{subsec:what-building}

\begin{svgraybox}
\textbf{Control Plane v0.1 Features:}
\begin{itemize}
\item \textbf{Single Backend Proxy}: Routes all requests to one Ollama instance
\item \textbf{OpenAI-Compatible API}: \texttt{/v1/completions} and \texttt{/v1/chat/completions} endpoints
\item \textbf{Health Endpoint}: \texttt{/health} returns backend status
\item \textbf{OpenTelemetry Observability}: Metrics, traces, and structured logs
\item \textbf{Structured Logging}: JSON logs with request IDs for tracing
\item \textbf{Configuration}: YAML file with environment variable overrides
\end{itemize}
\end{svgraybox}

\subsection{What We're Not Building (Yet)}
\label{subsec:not-building}

These features come in later chapters:
\begin{itemize}
\item Authentication and API keys (Chapter~\ref{ch:authentication})
\item Rate limiting and quotas (Chapter~\ref{ch:rate-limiting})
\item Response caching (Chapter~\ref{ch:caching})
\item Multiple backends and routing (Chapter~\ref{ch:model-routing})
\item Streaming responses (we add this in the problems section)
\end{itemize}

\subsection{Component Overview}
\label{subsec:components}

The control plane consists of five components:

\begin{description}
\item[API Server] HTTP server exposing OpenAI-compatible endpoints. Accepts requests, validates input, forwards to backend proxy.
\item[Backend Proxy] Translates OpenAI API format to backend-specific format (Ollama, vLLM). Handles response conversion.
\item[Health Checker] Periodically probes backend health. Exposes aggregated status via \texttt{/health}.
\item[Telemetry] Records metrics, traces, and logs via OpenTelemetry. Exports to configured backends (Grafana, Jaeger, etc.).
\item[Configuration] Loads settings from YAML and environment variables. Provides typed access to config values.
\end{description}

% =============================================================================
\section{Key Abstractions}
\label{sec:key-abstractions}
% =============================================================================

Good abstractions make code extensible. v0.1 defines two core interfaces that remain stable as we add features.

\subsection{The Backend Interface}
\label{subsec:backend-interface}

The \texttt{Backend} interface abstracts inference engines. Any engine that implements these three methods can plug into our control plane:

\begin{programcode}{Backend Interface (Go)}
\begin{lstlisting}[language=Go]
type Backend interface {
    // Generate performs inference and returns the response
    Generate(ctx context.Context, req GenerateRequest) (GenerateResponse, error)

    // Health checks if the backend is operational
    Health(ctx context.Context) error

    // Name returns the backend identifier for logging/metrics
    Name() string
}
\end{lstlisting}
\end{programcode}

Why these three methods?

\begin{description}
\item[Generate] The core operation. Takes a request, returns a response. The \texttt{context.Context} parameter enables timeouts and cancellation---essential for long-running inference.
\item[Health] Allows the control plane to detect backend failures before users do. Returns \texttt{nil} if healthy, error otherwise.
\item[Name] Used in metrics labels and logs. When you have multiple backends, you need to know which one handled each request.
\end{description}

The \texttt{GenerateRequest} and \texttt{GenerateResponse} types use OpenAI's format. This means:
\begin{itemize}
\item Clients can use existing OpenAI SDKs
\item Switching from OpenAI API to self-hosted requires only a URL change
\item Backend implementations translate to/from engine-specific formats
\end{itemize}

\subsection{The Health Checker Interface}
\label{subsec:health-interface}

Health checking is more nuanced than a simple up/down boolean:

\begin{programcode}{Health Status Types}
\begin{lstlisting}[language=Go]
type Status string

const (
    StatusHealthy   Status = "healthy"   // Fully operational
    StatusDegraded  Status = "degraded"  // Working but impaired
    StatusUnhealthy Status = "unhealthy" // Not operational
)

type CheckResult struct {
    Name    string  // Component name
    Status  Status  // Health status
    Message string  // Human-readable details
    Latency int64   // Check duration in milliseconds
}
\end{lstlisting}
\end{programcode}

The three-state model (\texttt{healthy}/\texttt{degraded}/\texttt{unhealthy}) enables smarter decisions than binary up/down. A degraded backend might still serve requests while you investigate, whereas unhealthy triggers alerts.

% =============================================================================
\section{Request Flow}
\label{sec:request-flow}
% =============================================================================

Understanding how requests flow through the system helps with debugging and optimization.

% Diagram: Request Flow through Control Plane
% Shows the step-by-step processing of a request

\begin{figure}[htbp]
\centering
\begin{tikzpicture}[
    node distance=0.6cm,
    flowstep/.style={rectangle, draw, rounded corners, minimum width=2.8cm, minimum height=0.7cm, align=center, font=\small},
    arrow/.style={->, thick, >=stealth},
    label/.style={font=\footnotesize\itshape, text=gray}
]

% Left column - Request path
\node[flowstep, fill=blue!15] (req) {1. Request\\Arrives};
\node[flowstep, fill=blue!10, below=of req] (log) {2. Logging\\Middleware};
\node[flowstep, fill=blue!10, below=of log] (metrics1) {3. Metrics\\Middleware};
\node[flowstep, fill=green!15, below=of metrics1] (validate) {4. Validate\\Request};
\node[flowstep, fill=orange!15, below=of validate] (backend) {5. Backend\\Call};

% Right column - Response path
\node[flowstep, fill=orange!15, right=3cm of backend] (process) {6. Process\\Response};
\node[flowstep, fill=purple!15, above=of process] (record) {7. Record\\Metrics};
\node[flowstep, fill=blue!15, above=of record] (resp) {8. Send\\Response};

% Arrows - down the left side
\draw[arrow] (req) -- (log);
\draw[arrow] (log) -- (metrics1);
\draw[arrow] (metrics1) -- (validate);
\draw[arrow] (validate) -- (backend);

% Arrow across to right side
\draw[arrow] (backend) -- node[above, label] {Ollama} (process);

% Arrows - up the right side
\draw[arrow] (process) -- (record);
\draw[arrow] (record) -- (resp);

% Side annotations
\node[label, anchor=east] at ($(req.west) - (0.3, 0)$) {POST /v1/completions};
\node[label, anchor=east] at ($(log.west) - (0.3, 0)$) {assign request\_id};
\node[label, anchor=east] at ($(metrics1.west) - (0.3, 0)$) {start timer};
\node[label, anchor=west] at ($(record.east) + (0.3, 0)$) {histogram, counter};
\node[label, anchor=west] at ($(resp.east) + (0.3, 0)$) {JSON response};

% Time indicator
\draw[<->, gray, dashed] ($(req.east) + (1.2, 0)$) -- ($(backend.east) + (1.2, 0)$);
\node[label, rotate=90] at ($(validate.east) + (1.5, 0)$) {~10ms overhead};

\draw[<->, gray, dashed] ($(backend.east) + (1.2, 0)$) -- ($(process.west) - (0.5, 0)$);
\node[label] at ($(backend.east) + (2.2, 0.3)$) {100ms-60s};
\node[label] at ($(backend.east) + (2.2, -0.1)$) {(inference)};

\end{tikzpicture}
\caption{Request flow through the control plane. Steps 1-4 add ~10ms overhead. Step 5 (inference) dominates latency. The request ID assigned in step 2 appears in all logs, enabling end-to-end tracing.}
\label{fig:request-flow}
\end{figure}


\subsection{Step-by-Step Flow}
\label{subsec:request-steps}

\begin{enumerate}
\item \textbf{Request Arrives}: Client sends POST to \texttt{/v1/completions} with JSON body
\item \textbf{Middleware Chain}: Request passes through logging middleware (assigns request ID), then metrics middleware (starts timer)
\item \textbf{Validation}: Handler validates required fields (model, prompt/messages)
\item \textbf{Backend Call}: Request is translated to backend format and forwarded
\item \textbf{Response Processing}: Backend response is translated to OpenAI format
\item \textbf{Metrics Recording}: Latency histogram updated, request counter incremented
\item \textbf{Response Sent}: JSON response returned to client
\end{enumerate}

Each step is instrumented. If something goes wrong, structured logs with the request ID let you trace the entire flow.

% =============================================================================
\section{Observability}
\label{sec:observability}
% =============================================================================

You can't improve what you can't measure. v0.1 includes observability from day one using OpenTelemetry.

\subsection{Why OpenTelemetry}
\label{subsec:why-otel}

OpenTelemetry (OTel) is the industry standard for observability, providing unified APIs for metrics, traces, and logs. We use it throughout this book because:

\begin{description}
\item[Vendor Neutral] Export to any backend---Prometheus, Grafana, Jaeger, Datadog, or cloud-native options. Switch providers without changing code.
\item[Unified SDK] One library for metrics, traces, and logs. Correlation between them is automatic.
\item[Distributed Tracing] Essential as we add multiple backends, caching, and queues in later chapters. Trace requests across services.
\item[Industry Standard] Supported by all major cloud providers and observability vendors. Skills transfer everywhere.
\end{description}

\subsection{Metrics That Matter}
\label{subsec:metrics-matter}

We track four categories of metrics:

\begin{description}
\item[Request Volume] \texttt{inference.requests.total} --- Counter of requests by model and status (success/error). Answers: How busy are we? What's the error rate?

\item[Latency] \texttt{inference.request.duration} --- Histogram of request duration. Answers: What's the p50/p95/p99 latency? Are we meeting SLOs?

\item[Throughput] \texttt{inference.tokens.generated} --- Counter of tokens generated. Answers: What's our actual output? (More meaningful than request count for LLMs.)

\item[Backend Health] \texttt{inference.backend.healthy} --- Gauge (1 = healthy, 0 = unhealthy). Answers: Is our backend up? (Useful for alerting.)
\end{description}

\subsection{Distributed Tracing}
\label{subsec:distributed-tracing}

Every request gets a trace ID that follows it through the system. In v0.1 this seems like overkill---there's only one backend. But as we add caching, queues, and multiple backends, traces become essential for debugging.

Each trace contains spans showing:
\begin{itemize}
\item Time spent in middleware (auth, rate limiting)
\item Time spent waiting in queue
\item Time spent in backend inference
\item Time spent in response processing
\end{itemize}

\subsection{Why Histograms for Latency}
\label{subsec:why-histograms}

LLM inference latency varies wildly based on prompt length, output length, and model load. A simple average hides important details. Histograms let you compute arbitrary percentiles:

\begin{itemize}
\item \textbf{p50}: Typical user experience
\item \textbf{p95}: Experience for most users
\item \textbf{p99}: Worst case (tail latency)
\end{itemize}

Our histogram buckets are tuned for inference: 0.1s, 0.5s, 1s, 2s, 5s, 10s, 30s, 60s. Short requests (cached, small output) hit early buckets; long generations hit later ones.

\subsection{Structured Logging}
\label{subsec:structured-logging}

Every log entry includes:
\begin{itemize}
\item \texttt{trace\_id}: OpenTelemetry trace ID for correlation
\item \texttt{span\_id}: Current span within the trace
\item \texttt{method}: HTTP method
\item \texttt{path}: Request path
\item \texttt{status}: Response status code
\item \texttt{duration\_ms}: Request duration
\item \texttt{model}: Model name (when applicable)
\end{itemize}

JSON-formatted logs with trace context can be shipped to any log aggregator (Loki, Elasticsearch, CloudWatch) and correlated with traces.

% =============================================================================
\section{Deployment}
\label{sec:deployment}
% =============================================================================

v0.1 deploys as a Docker Compose stack with four services. From this chapter onward, we recommend deploying to a cloud GPU provider for a consistent environment regardless of your local hardware.

\subsection{Why Cloud GPUs?}
\label{subsec:why-cloud}

Running inference locally works for experimentation, but cloud GPUs provide:
\begin{itemize}
\item \textbf{Consistent Environment}: Same hardware, same results for all readers
\item \textbf{Pay-per-Use}: No upfront investment in expensive hardware
\item \textbf{Instant Scaling}: Spin up larger GPUs when needed
\item \textbf{Production-like}: Practice deploying to real infrastructure
\end{itemize}

Appendix~\ref{ch:appendix-gpu-providers} provides detailed setup guides for AWS, GCP, Lambda Labs, RunPod, and Vast.ai. For Part I of this book, any provider with an NVIDIA GPU (16GB+ VRAM) works.

\subsection{Deployment Topology}
\label{subsec:deployment-topology}

\begin{center}
\begin{tabular}{llll}
\toprule
\textbf{Service} & \textbf{Port} & \textbf{Purpose} \\
\midrule
control-plane & 8080 & API server (your application calls this) \\
ollama & 11434 & Inference engine (control plane calls this) \\
otel-collector & 4317 & OpenTelemetry collector (receives OTLP) \\
grafana & 3000 & Dashboards, metrics, traces, logs \\
\bottomrule
\end{tabular}
\end{center}

The control plane is the only service your application needs to know about. It exports telemetry via OTLP to the OpenTelemetry Collector, which routes metrics to Prometheus, traces to Tempo, and logs to Loki---all visualized in Grafana.

\subsection{Resource Requirements}
\label{subsec:resources}

For running a 7B model with v0.1:
\begin{itemize}
\item \textbf{Control Plane}: 128MB RAM, minimal CPU (it's just proxying)
\item \textbf{Ollama}: 8-16GB VRAM (GPU), or 16GB+ RAM (CPU, slower)
\item \textbf{OTel Collector + Backends}: 512MB RAM (Grafana stack)
\end{itemize}

\begin{svgraybox}
\textbf{Recommended Cloud Instance (Part I):}

A single instance with 24GB GPU (NVIDIA A10 or RTX 4090) is sufficient for all Part I exercises. Approximate cost: \$0.50-1.00/hour on budget providers. See Appendix~\ref{ch:appendix-gpu-providers} for provider-specific recommendations.
\end{svgraybox}

% =============================================================================
\section{Running the Control Plane}
\label{sec:running}
% =============================================================================

The complete implementation is available at:

\begin{center}
\texttt{github.com/inference-book/inference-plane}
\end{center}

\subsection{Quick Start}
\label{subsec:quickstart}

\begin{lstlisting}[language=bash]
# Clone the repository
git clone https://github.com/inference-book/inference-plane
cd inference-plane

# Checkout v0.1
git checkout v0.1

# Start the stack
docker compose up -d

# Pull a model (first time only)
docker compose exec ollama ollama pull llama3.2:7b

# Verify it's working
curl http://localhost:8080/health
\end{lstlisting}

\subsection{Making Requests}
\label{subsec:making-requests}

The control plane exposes an OpenAI-compatible API:

\begin{programcode}{Completion Request}
\begin{lstlisting}[language=bash]
curl -X POST http://localhost:8080/v1/completions \
  -H "Content-Type: application/json" \
  -d '{
    "model": "llama3.2:7b",
    "prompt": "Explain inference in one sentence:",
    "max_tokens": 50
  }'
\end{lstlisting}
\end{programcode}

Response follows OpenAI format:

\begin{lstlisting}[language=bash]
{
  "id": "gen-1234567890",
  "object": "text_completion",
  "created": 1703001234,
  "model": "llama3.2:7b",
  "choices": [{
    "index": 0,
    "text": "Inference is the process of using a trained model...",
    "finish_reason": "stop"
  }],
  "usage": {
    "prompt_tokens": 8,
    "completion_tokens": 42,
    "total_tokens": 50
  }
}
\end{lstlisting}

\subsection{Viewing Telemetry}
\label{subsec:viewing-telemetry}

After making some requests, explore your observability data:

\begin{lstlisting}[language=bash]
# Open Grafana (includes dashboards for metrics, traces, logs)
open http://localhost:3000  # admin/admin

# Pre-configured dashboards:
# - Inference Overview: request rate, latency, errors
# - Traces: distributed request tracing
# - Logs: structured log exploration
\end{lstlisting}

The repository includes pre-configured Grafana dashboards and datasources for the full observability stack.

% =============================================================================
\section{Implementation Notes}
\label{sec:implementation-notes}
% =============================================================================

A few design decisions worth highlighting:

\subsection{Why Go?}
\label{subsec:why-go}

\begin{itemize}
\item \textbf{Single Binary}: No runtime dependencies, easy deployment
\item \textbf{Fast Startup}: Sub-second cold start (important for scaling)
\item \textbf{Low Memory}: Control plane uses ~20MB RSS
\item \textbf{Excellent HTTP}: Standard library handles HTTP efficiently
\item \textbf{Concurrency}: Goroutines handle many concurrent requests naturally
\end{itemize}

The patterns in this book translate to other languages. A Python (FastAPI) or Rust (Axum) implementation would use the same abstractions.

\subsection{Configuration Hierarchy}
\label{subsec:config-hierarchy}

Configuration loads in order (later overrides earlier):
\begin{enumerate}
\item Compiled defaults
\item YAML config file (\texttt{config.yaml})
\item Environment variables (prefixed with \texttt{CP\_})
\end{enumerate}

This allows: sensible defaults for development, file-based config for staging, environment variables for production (12-factor app style).

\subsection{Error Handling}
\label{subsec:error-handling}

Backend errors are translated to appropriate HTTP status codes:
\begin{itemize}
\item Backend timeout → 504 Gateway Timeout
\item Backend unavailable → 503 Service Unavailable
\item Invalid request → 400 Bad Request
\item Backend error → 502 Bad Gateway
\end{itemize}

Error responses include a structured JSON body with error details, making client-side handling straightforward.

% =============================================================================
\section{What's Next}
\label{sec:whats-next}
% =============================================================================

v0.1 is a foundation. Here's how it evolves:

\begin{svgraybox}
\textbf{Control Plane Evolution:}
\begin{description}
\item[v0.1 (This Chapter)] Single backend, metrics, health checks
\item[v0.2 (Part II)] Authentication, rate limiting, caching
\item[v0.3 (Part III)] Multi-tenant, billing, multiple backends
\item[v1.0 (Part IV)] Production-hardened, distributed, auto-scaling
\end{description}
\end{svgraybox}

Each version builds on the previous. The \texttt{Backend} interface you learned here remains unchanged---new backends (vLLM, TensorRT-LLM) simply implement the same interface.

% =============================================================================
\section{Summary}
\label{sec:ch06-summary}
% =============================================================================

\begin{important}{Key Takeaways}
\begin{itemize}
\item A control plane separates policy (how requests are handled) from computation (inference)
\item The \texttt{Backend} interface abstracts inference engines, enabling swappable backends
\item OpenAI-compatible API means existing tools and SDKs work immediately
\item Observability via OpenTelemetry (metrics, traces, logs) is built in from day one
\item v0.1 is minimal by design---complexity is added incrementally in later chapters
\end{itemize}
\end{important}

\begin{svgraybox}
\textbf{Repository Reference:}

The complete v0.1 implementation, including Dockerfile, Docker Compose, Grafana dashboards, and tests, is available at:

\texttt{github.com/inference-book/inference-plane}

Clone it, run it, modify it. The code is Apache 2.0 licensed.
\end{svgraybox}

% =============================================================================
\section*{Problems}
\addcontentsline{toc}{section}{Problems}
% =============================================================================

\begin{prob}
\label{prob:ch06-vllm-backend}
\textbf{vLLM Backend Implementation}\\
Implement a backend that connects to vLLM's OpenAI-compatible API. Since vLLM already speaks OpenAI format, this should be simpler than the Ollama backend. What translation is still needed?
\end{prob}

\begin{prob}
\label{prob:ch06-streaming}
\textbf{Streaming Support}\\
Extend the control plane to support streaming responses (Server-Sent Events). You'll need to: (a) detect when the client wants streaming, (b) proxy the stream from the backend, (c) update metrics to handle streaming responses.
\end{prob}

\begin{prob}
\label{prob:ch06-multi-model}
\textbf{Model Routing (Preview)}\\
Modify the control plane to support multiple models on the same Ollama backend. The request's \texttt{model} field should be passed through to Ollama. Add a metric label for model name.
\end{prob}

\begin{prob}
\label{prob:ch06-dashboard}
\textbf{Custom Dashboard}\\
Create a Grafana dashboard that shows: (a) requests per minute over time, (b) p50/p95/p99 latency, (c) tokens per second, (d) error rate percentage. Export as JSON and submit.
\end{prob}

%%%%%%%%%%%%%%%%%%%%%%%% references02.tex %%%%%%%%%%%%%%%%%%%%%%%%%%%%%%
% References for Chapter 2
%%%%%%%%%%%%%%%%%%%%%%%% Springer-Verlag %%%%%%%%%%%%%%%%%%%%%%%%%%

\begin{thebibliography}{99.}

\bibitem{nvidia2024h100} NVIDIA: H100 Tensor Core GPU Datasheet. \url{https://www.nvidia.com/en-us/data-center/h100/} (2024)

\bibitem{nvidia2024a100} NVIDIA: A100 Tensor Core GPU Datasheet. \url{https://www.nvidia.com/en-us/data-center/a100/} (2024)

\bibitem{amd2024mi300} AMD: Instinct MI300X Accelerator. \url{https://www.amd.com/en/products/accelerators/instinct/mi300/mi300x.html} (2024)

\bibitem{apple2024m4} Apple: M4 Chip Architecture. \url{https://www.apple.com/newsroom/2024/05/apple-introduces-m4-chip/} (2024)

\bibitem{runpod2025pricing} RunPod: GPU Cloud Pricing. \url{https://www.runpod.io/gpu-instance/pricing} (accessed December 2025)

\bibitem{vastai2025} Vast.ai: GPU Marketplace. \url{https://vast.ai/} (accessed December 2025)

\bibitem{lambdalabs2025} Lambda Labs: GPU Cloud. \url{https://lambdalabs.com/service/gpu-cloud} (accessed December 2025)

\end{thebibliography}

