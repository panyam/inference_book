%%%%%%%%%%%%%%%%%%%%% chapter02.tex %%%%%%%%%%%%%%%%%%%%%%%%%%%%%%%%%
%
% Chapter 2: Hardware Fundamentals
%
%%%%%%%%%%%%%%%%%%%%%%%% Springer-Verlag %%%%%%%%%%%%%%%%%%%%%%%%%%

\chapter{Hardware Fundamentals}
\label{ch:hardware}

\abstract*{Understanding hardware is essential for making informed decisions about self-hosted inference. This chapter covers how transformers use memory during inference, VRAM calculation formulas, CPU vs GPU trade-offs, and a comprehensive comparison of available hardware options. We also analyze the economics of owned versus rented infrastructure to help you make the right choice for your situation.}

\abstract{Understanding hardware is essential for making informed decisions about self-hosted inference. This chapter covers how transformers use memory during inference, VRAM calculation formulas, CPU vs GPU trade-offs, and a comprehensive comparison of available hardware options. We also analyze the economics of owned versus rented infrastructure to help you make the right choice for your situation.}

% =============================================================================
\section{How Transformers Use Memory}
\label{sec:transformer-memory}
% =============================================================================

% TODO: Explain memory usage during inference
% - Model weights
% - KV cache
% - Activation memory
% - Working memory

\subsection{Model Weights}
\label{subsec:model-weights}

% TODO: Weight storage
% - Parameters and precision
% - Memory formula: params × bytes_per_param
% - Example calculations

\subsection{The KV Cache}
\label{subsec:kv-cache}

% TODO: Explain KV cache
% - What it stores
% - Why it grows with context
% - Memory formula for KV cache
% - Why it matters for long contexts

\subsection{Activation and Working Memory}
\label{subsec:activation-memory}

% TODO: Other memory needs
% - Intermediate activations
% - Batch processing overhead
% - Safety margins

% =============================================================================
\section{VRAM Calculation}
\label{sec:vram-calculation}
% =============================================================================

% TODO: The master formula

\begin{svgraybox}
\textbf{VRAM Estimation Formula:}

For a model with $P$ parameters at precision $B$ bytes:

\begin{equation}
\text{VRAM}_{\text{base}} = P \times B
\end{equation}

For inference with context length $C$ and batch size $N$:

\begin{equation}
\text{VRAM}_{\text{total}} \approx P \times B + \text{KV\_cache}(C, N) + \text{overhead}
\end{equation}

Where overhead is typically 10--20\% additional memory.
\end{svgraybox}

\subsection{Precision and Memory}
\label{subsec:precision}

% TODO: Table of precision formats
% | Precision | Bytes | Use Case |
% | FP32 | 4 | Training (rarely inference) |
% | FP16 | 2 | Standard GPU inference |
% | BF16 | 2 | Standard GPU inference |
% | INT8 | 1 | Quantized inference |
% | INT4 | 0.5 | Aggressive quantization |

\subsection{Worked Examples}
\label{subsec:vram-examples}

% TODO: Calculate for our target models
% - 7B model at different precisions
% - 30B model requirements
% - 70B model requirements
% - 400B model requirements

% =============================================================================
\section{CPU vs GPU Inference}
\label{sec:cpu-vs-gpu}
% =============================================================================

% TODO: When each makes sense

\subsection{CPU Inference}
\label{subsec:cpu-inference}

% TODO: CPU characteristics
% - Larger memory (RAM vs VRAM)
% - Lower throughput
% - llama.cpp optimization
% - When it works well

\subsection{GPU Inference}
\label{subsec:gpu-inference}

% TODO: GPU advantages
% - Parallelism
% - Memory bandwidth
% - Specialized tensor cores
% - When it's required

\subsection{Hybrid Approaches}
\label{subsec:hybrid-cpu-gpu}

% TODO: CPU+GPU together
% - Offloading layers
% - llama.cpp layer splitting
% - Trade-offs

% =============================================================================
\section{GPU Hardware Comparison}
\label{sec:gpu-comparison}
% =============================================================================

% TODO: Comprehensive GPU comparison

\subsection{NVIDIA Consumer GPUs}
\label{subsec:nvidia-consumer}

% TODO: RTX series comparison
% - RTX 3060 12GB
% - RTX 3090 24GB
% - RTX 4060 8GB
% - RTX 4070 Ti 16GB
% - RTX 4080 16GB
% - RTX 4090 24GB

\subsection{NVIDIA Professional GPUs}
\label{subsec:nvidia-professional}

% TODO: Workstation and datacenter
% - RTX A4000/A5000/A6000
% - A100 40GB/80GB
% - H100 80GB
% - Comparison table

\subsection{AMD GPUs}
\label{subsec:amd-gpus}

% TODO: AMD options
% - ROCm support status
% - RX 7900 XTX
% - Instinct MI series
% - Trade-offs vs NVIDIA

\subsection{Apple Silicon}
\label{subsec:apple-silicon}

% TODO: M-series chips
% - Unified memory advantage
% - M1/M2/M3 performance
% - M1/M2/M3 Pro/Max/Ultra
% - When they make sense

\subsection{Hardware Recommendation Matrix}
\label{subsec:hardware-matrix}

% TODO: Decision matrix table
% | Model Size | Min VRAM | Recommended | Budget Option |
% | 7B Q4 | 4GB | RTX 4060 8GB | M1 MacBook |
% | 7B FP16 | 14GB | RTX 4090 24GB | RTX 3090 24GB |
% | etc. |

% =============================================================================
\section{Owned vs Rented Infrastructure}
\label{sec:owned-vs-rented}
% =============================================================================

% TODO: The big economic question

\subsection{Buying Hardware}
\label{subsec:buying-hardware}

% TODO: Ownership economics
% - Upfront CapEx
% - Ongoing costs (power, cooling, maintenance)
% - Depreciation
% - Tax implications
% - When it makes sense

\subsection{Renting Cloud GPUs}
\label{subsec:renting-gpus}

% TODO: Rental options
% - RunPod
% - Vast.ai
% - Lambda Labs
% - Major cloud providers (AWS, GCP, Azure)
% - Pricing comparison

\subsection{Break-Even Analysis}
\label{subsec:break-even}

% TODO: When to buy vs rent
% - Formula for break-even
% - Worked examples
% - Utilization factor
% - Decision framework

\begin{svgraybox}
\textbf{Break-Even Formula:}

\begin{equation}
\text{Break-even (months)} = \frac{\text{Hardware Cost} + \text{Setup Costs}}{\text{Monthly Rental Cost} - \text{Monthly Operating Cost}}
\end{equation}

Where Monthly Operating Cost includes power, cooling, and amortized maintenance.
\end{svgraybox}

\subsection{Hybrid Strategies}
\label{subsec:hybrid-strategies}

% TODO: Best of both worlds
% - Own base capacity
% - Rent for burst
% - Geographic distribution

% =============================================================================
\section{Power and Cooling Considerations}
\label{sec:power-cooling}
% =============================================================================

% TODO: Practical infrastructure
% - Power consumption by GPU
% - Cooling requirements
% - Home lab considerations
% - Data center placement

% =============================================================================
\section{TPU as an Alternative}
\label{sec:tpu-sidebar}
% =============================================================================

\begin{backgroundinformation}{Google Cloud TPUs}
While this book primarily focuses on GPU-based inference, Google Cloud TPUs offer an alternative worth considering. TPUs can be 50--70\% cheaper than equivalent GPUs for sustained workloads, but require JAX-based models and are only available on Google Cloud Platform.

If you're deploying on GCP with JAX models, see Appendix~\ref{ch:appendix-tpu} for complete coverage of TPU inference.
\end{backgroundinformation}

% =============================================================================
\section{Summary}
\label{sec:ch02-summary}
% =============================================================================

% TODO: Chapter summary

\begin{important}{Key Takeaways}
\begin{itemize}
\item VRAM is the primary constraint for GPU inference
\item Quantization dramatically reduces memory requirements (4x for INT4)
\item CPU inference is viable for smaller models with enough RAM
\item Hardware choice depends on model size, budget, and utilization
\item Break-even between owned and rented depends on consistent usage
\end{itemize}
\end{important}

% =============================================================================
\section*{Problems}
\addcontentsline{toc}{section}{Problems}
% =============================================================================

\begin{prob}
\label{prob:ch02-vram-calc}
\textbf{VRAM Calculation}\\
Calculate the minimum VRAM required to run a 30B parameter model at FP16 precision with a context length of 4096 tokens. Then calculate the VRAM needed if quantized to INT4.
\end{prob}

\begin{prob}
\label{prob:ch02-break-even}
\textbf{Break-Even Analysis}\\
You need to serve a 7B model 24/7. Calculate the break-even point (in months) between buying an RTX 4070 Ti (\$800) versus renting an equivalent GPU on RunPod (\$0.30/hour). Assume \$0.15/kWh electricity and 200W GPU power consumption.
\end{prob}

\begin{prob}
\label{prob:ch02-hardware-selection}
\textbf{Hardware Selection}\\
A startup needs to serve 1000 requests/hour with a 7B model. They have a budget of \$5,000. Recommend a hardware configuration and justify your choice, considering both owned and rented options.
\end{prob}

% =============================================================================
% References for Chapter 2: How LLMs Work
% =============================================================================

\begin{thebibliography}{99}

\bibitem{vaswani2017attention}
Vaswani, A., et al.: Attention Is All You Need. In: Advances in Neural Information Processing Systems (NeurIPS) 2017. \url{https://arxiv.org/abs/1706.03762}

\bibitem{radford2018gpt}
Radford, A., et al.: Improving Language Understanding by Generative Pre-Training. OpenAI (2018). \url{https://openai.com/research/language-unsupervised}

\bibitem{jiang2024mixtral}
Jiang, A., et al.: Mixtral of Experts. Mistral AI (2024). \url{https://arxiv.org/abs/2401.04088}

\bibitem{gu2023mamba}
Gu, A., Dao, T.: Mamba: Linear-Time Sequence Modeling with Selective State Spaces. (2023). \url{https://arxiv.org/abs/2312.00752}

\bibitem{pope2022efficiently}
Pope, R., et al.: Efficiently Scaling Transformer Inference. MLSys 2023. \url{https://arxiv.org/abs/2211.05102}

\end{thebibliography}

