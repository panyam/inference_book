%%%%%%%%%%%%%%%%%%%%% chapter09.tex %%%%%%%%%%%%%%%%%%%%%%%%%%%%%%%%%
%
% Chapter 9: Request Queue and Priority Management
%
%%%%%%%%%%%%%%%%%%%%%%%% Springer-Verlag %%%%%%%%%%%%%%%%%%%%%%%%%%

\chapter{Request Queue and Priority Management}
\label{ch:queue}

\abstract*{When demand exceeds capacity, you need a queue. This chapter covers request queue design, priority levels, fairness algorithms, timeout handling, and queue-aware load shedding. We implement a priority queue system that ensures important requests get processed while preventing queue explosion during traffic spikes.}

\abstract{When demand exceeds capacity, you need a queue. This chapter covers request queue design, priority levels, fairness algorithms, timeout handling, and queue-aware load shedding. We implement a priority queue system that ensures important requests get processed while preventing queue explosion during traffic spikes.}

% =============================================================================
\section{Why Queue Requests?}
\label{sec:why-queue}
% =============================================================================

% TODO: Motivation
% - Traffic spikes
% - GPU utilization optimization
% - Fair resource allocation
% - Graceful overload handling

% =============================================================================
\section{Queue Architecture}
\label{sec:queue-architecture}
% =============================================================================

% TODO: Design considerations

\subsection{In-Memory vs Distributed}
\label{subsec:queue-types}

% TODO: Trade-offs
% - In-memory: simple, fast, single node
% - Distributed: Redis, RabbitMQ, etc.

\subsection{Priority Levels}
\label{subsec:priority-levels}

% TODO: Defining priorities
% - Enterprise (highest)
% - Pro
% - Free (lowest)
% - System/internal

\begin{programcode}{Priority Queue Interface}
\begin{lstlisting}[language=Go]
// internal/queue/queue.go

package queue

import (
    "context"
    "time"
)

type Priority int

const (
    PriorityLow Priority = iota
    PriorityNormal
    PriorityHigh
    PriorityCritical
)

type QueuedRequest struct {
    ID          string
    UserID      string
    Priority    Priority
    Request     backend.GenerateRequest
    EnqueuedAt  time.Time
    Deadline    time.Time
    ResultChan  chan QueueResult
}

type QueueResult struct {
    Response backend.GenerateResponse
    Error    error
    WaitTime time.Duration
}

type RequestQueue interface {
    // Enqueue adds a request to the queue
    Enqueue(ctx context.Context, req QueuedRequest) error

    // Dequeue gets the next request to process
    Dequeue(ctx context.Context) (*QueuedRequest, error)

    // Size returns current queue length by priority
    Size(ctx context.Context) map[Priority]int

    // Stats returns queue statistics
    Stats(ctx context.Context) QueueStats
}

type QueueStats struct {
    TotalEnqueued   int64
    TotalProcessed  int64
    TotalTimeout    int64
    AverageWaitTime time.Duration
    CurrentDepth    int
}
\end{lstlisting}
\end{programcode}

% =============================================================================
\section{Implementing Priority Queue}
\label{sec:priority-queue-impl}
% =============================================================================

% TODO: Implementation

\subsection{Heap-Based Priority Queue}
\label{subsec:heap-queue}

% TODO: Using container/heap
% - Standard library approach
% - Efficient O(log n) operations

\subsection{Multiple Queue Approach}
\label{subsec:multi-queue}

% TODO: Separate queues per priority
% - Simple round-robin
% - Weighted fair queueing

% =============================================================================
\section{Fairness Algorithms}
\label{sec:fairness}
% =============================================================================

% TODO: Preventing starvation

\subsection{Weighted Fair Queueing}
\label{subsec:wfq}

% TODO: WFQ explained
% - Weights per priority
% - Preventing low-priority starvation

\subsection{Deficit Round Robin}
\label{subsec:drr}

% TODO: DRR for simplicity
% - How it works
% - Implementation

% =============================================================================
\section{Timeout Handling}
\label{sec:timeout-handling}
% =============================================================================

% TODO: Managing deadlines

\subsection{Queue Timeout}
\label{subsec:queue-timeout}

% TODO: Max time in queue
% - Per-priority timeouts
% - Removing stale requests

\subsection{Processing Timeout}
\label{subsec:processing-timeout}

% TODO: Max inference time
% - Context cancellation
% - Resource cleanup

% =============================================================================
\section{Load Shedding}
\label{sec:load-shedding}
% =============================================================================

% TODO: When queue gets too long

\subsection{Queue Limits}
\label{subsec:queue-limits}

% TODO: Max queue depth
% - Per-priority limits
% - System-wide limits

\subsection{Graceful Rejection}
\label{subsec:graceful-rejection}

% TODO: How to reject
% - 503 Service Unavailable
% - Retry-After header
% - Queue position estimation

\begin{programcode}{Load Shedding Configuration}
\begin{lstlisting}[language=Go]
// internal/queue/config.go

type QueueConfig struct {
    // Maximum queue depth per priority
    MaxDepth map[Priority]int `yaml:"max_depth"`

    // Maximum wait time in queue
    MaxWaitTime map[Priority]time.Duration `yaml:"max_wait_time"`

    // Processing timeout
    ProcessingTimeout time.Duration `yaml:"processing_timeout"`

    // Weights for fair queueing (higher = more service)
    Weights map[Priority]int `yaml:"weights"`
}

var DefaultQueueConfig = QueueConfig{
    MaxDepth: map[Priority]int{
        PriorityLow:      100,
        PriorityNormal:   500,
        PriorityHigh:     1000,
        PriorityCritical: 5000,
    },
    MaxWaitTime: map[Priority]time.Duration{
        PriorityLow:      30 * time.Second,
        PriorityNormal:   60 * time.Second,
        PriorityHigh:     120 * time.Second,
        PriorityCritical: 300 * time.Second,
    },
    ProcessingTimeout: 5 * time.Minute,
    Weights: map[Priority]int{
        PriorityLow:      1,
        PriorityNormal:   3,
        PriorityHigh:     10,
        PriorityCritical: 50,
    },
}
\end{lstlisting}
\end{programcode}

% =============================================================================
\section{Worker Pool}
\label{sec:worker-pool}
% =============================================================================

% TODO: Processing requests from queue

\subsection{Worker Design}
\label{subsec:worker-design}

% TODO: Worker goroutines
% - Pool size configuration
% - Graceful shutdown

\subsection{Scaling Workers}
\label{subsec:scaling-workers}

% TODO: Dynamic scaling
% - Based on queue depth
% - Based on GPU utilization

% =============================================================================
\section{Metrics and Visibility}
\label{sec:queue-metrics}
% =============================================================================

% TODO: What to track
% - Queue depth
% - Wait time percentiles
% - Rejection rate
% - Timeout rate

% =============================================================================
\section{Integration with Control Plane}
\label{sec:queue-integration}
% =============================================================================

% TODO: Wiring it together
% - Synchronous vs async API
% - Webhook callbacks
% - Polling endpoints

% =============================================================================
\section{Summary}
\label{sec:ch09-summary}
% =============================================================================

\begin{important}{Key Takeaways}
\begin{itemize}
\item Queues enable graceful handling of traffic spikes
\item Priority levels ensure important requests get served first
\item Fairness algorithms prevent starvation of lower priorities
\item Load shedding protects the system from overload
\item Clear visibility into queue state is essential
\end{itemize}
\end{important}

% =============================================================================
\section*{Problems}
\addcontentsline{toc}{section}{Problems}
% =============================================================================

\begin{prob}
\label{prob:ch09-priority-queue}
\textbf{Priority Queue Implementation}\\
Implement a priority queue using Go's container/heap package. Support multiple priority levels with configurable weights.
\end{prob}

\begin{prob}
\label{prob:ch09-worker-pool}
\textbf{Worker Pool}\\
Build a worker pool that processes requests from the priority queue. Include graceful shutdown and dynamic scaling based on queue depth.
\end{prob}

\begin{prob}
\label{prob:ch09-load-test}
\textbf{Load Testing}\\
Create a load testing tool that generates traffic at various rates. Measure queue behavior under normal load, overload, and spike scenarios.
\end{prob}

%%%%%%%%%%%%%%%%%%%%%%%% references08.tex %%%%%%%%%%%%%%%%%%%%%%%%%%%%%%
% References for Chapter 8
%%%%%%%%%%%%%%%%%%%%%%%% Springer-Verlag %%%%%%%%%%%%%%%%%%%%%%%%%%

\begin{thebibliography}{99.}

% TODO: Add references as chapter is written

\end{thebibliography}

