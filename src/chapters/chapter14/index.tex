%%%%%%%%%%%%%%%%%%%%% chapter13.tex %%%%%%%%%%%%%%%%%%%%%%%%%%%%%%%%%
%
% Chapter 13: Multi-GPU and Distributed Inference
%
%%%%%%%%%%%%%%%%%%%%%%%% Springer-Verlag %%%%%%%%%%%%%%%%%%%%%%%%%%

\chapter{Multi-GPU and Distributed Inference}
\label{ch:distributed}

\abstract*{70B models don't fit on a single consumer GPU. This chapter covers tensor parallelism, pipeline parallelism, NVLink considerations, and orchestrating inference across multiple GPUs. We configure vLLM for multi-GPU inference and integrate distributed backends into our control plane.}

\abstract{70B models don't fit on a single consumer GPU. This chapter covers tensor parallelism, pipeline parallelism, NVLink considerations, and orchestrating inference across multiple GPUs. We configure vLLM for multi-GPU inference and integrate distributed backends into our control plane.}

% =============================================================================
\section{Why Distributed Inference?}
\label{sec:why-distributed}
% =============================================================================

% TODO: Motivation
% - Model too large for single GPU
% - Higher throughput
% - Redundancy

% =============================================================================
\section{Parallelism Strategies}
\label{sec:parallelism}
% =============================================================================

\subsection{Tensor Parallelism}
\label{subsec:tensor-parallel}

% TODO: Splitting layers across GPUs
% - How it works
% - Communication overhead
% - NVLink importance

\subsection{Pipeline Parallelism}
\label{subsec:pipeline-parallel}

% TODO: Sequential layer groups
% - Less communication
% - Bubble overhead

\subsection{Data Parallelism}
\label{subsec:data-parallel}

% TODO: Separate replicas
% - Simplest approach
% - Good for throughput

% =============================================================================
\section{Hardware Topology}
\label{sec:topology}
% =============================================================================

% TODO: Understanding GPU connections
% - NVLink
% - PCIe
% - Network (multi-node)

% =============================================================================
\section{Configuring vLLM for Multi-GPU}
\label{sec:vllm-multigpu}
% =============================================================================

\begin{programcode}{Multi-GPU vLLM Configuration}
\begin{lstlisting}[language=bash]
# 70B model across 2 GPUs
python -m vllm.entrypoints.openai.api_server \
    --model meta-llama/Llama-3.1-70B-Instruct \
    --tensor-parallel-size 2 \
    --dtype float16 \
    --max-model-len 4096 \
    --gpu-memory-utilization 0.90
\end{lstlisting}
\end{programcode}

% =============================================================================
\section{Multi-Node Deployment}
\label{sec:multi-node}
% =============================================================================

% TODO: Across physical machines
% - Ray for orchestration
% - Network considerations

% =============================================================================
\section{Control Plane Integration}
\label{sec:distributed-cp}
% =============================================================================

% TODO: Managing distributed backends
% - Health across all GPUs
% - Routing to available replicas

% =============================================================================
\section{Summary}
\label{sec:ch13-summary}
% =============================================================================

\begin{important}{Key Takeaways}
\begin{itemize}
\item Tensor parallelism splits model layers across GPUs
\item NVLink significantly improves multi-GPU performance
\item vLLM handles parallelism transparently
\item Control plane needs awareness of distributed backends
\end{itemize}
\end{important}

\section*{Problems}
\addcontentsline{toc}{section}{Problems}

\begin{prob}
\label{prob:ch13-benchmark}
Benchmark a 70B model with tensor parallel sizes of 2 and 4. Measure throughput and latency differences.
\end{prob}

%%%%%%%%%%%%%%%%%%%%%%%% references02.tex %%%%%%%%%%%%%%%%%%%%%%%%%%%%%%
% References for Chapter 2
%%%%%%%%%%%%%%%%%%%%%%%% Springer-Verlag %%%%%%%%%%%%%%%%%%%%%%%%%%

\begin{thebibliography}{99.}

\bibitem{nvidia2024h100} NVIDIA: H100 Tensor Core GPU Datasheet. \url{https://www.nvidia.com/en-us/data-center/h100/} (2024)

\bibitem{nvidia2024a100} NVIDIA: A100 Tensor Core GPU Datasheet. \url{https://www.nvidia.com/en-us/data-center/a100/} (2024)

\bibitem{amd2024mi300} AMD: Instinct MI300X Accelerator. \url{https://www.amd.com/en/products/accelerators/instinct/mi300/mi300x.html} (2024)

\bibitem{apple2024m4} Apple: M4 Chip Architecture. \url{https://www.apple.com/newsroom/2024/05/apple-introduces-m4-chip/} (2024)

\bibitem{runpod2025pricing} RunPod: GPU Cloud Pricing. \url{https://www.runpod.io/gpu-instance/pricing} (accessed December 2025)

\bibitem{vastai2025} Vast.ai: GPU Marketplace. \url{https://vast.ai/} (accessed December 2025)

\bibitem{lambdalabs2025} Lambda Labs: GPU Cloud. \url{https://lambdalabs.com/service/gpu-cloud} (accessed December 2025)

\end{thebibliography}

