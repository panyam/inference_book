%%%%%%%%%%%%%%%%%%%%% chapter05_5.tex %%%%%%%%%%%%%%%%%%%%%%%%%%%%%%%%%
%
% Chapter 5.5: Edge Inference - Browser AI & Hybrid Architecture
%
%%%%%%%%%%%%%%%%%%%%%%%% Springer-Verlag %%%%%%%%%%%%%%%%%%%%%%%%%%

\chapter{Edge Inference: Browser AI and Hybrid Architecture}
\label{ch:browser-ai}

\abstract*{This optional chapter explores browser-based inference as a way to achieve zero marginal cost at scale. We cover WebGPU and WebGL technologies, browser inference frameworks (WebLLM, Transformers.js), hybrid client-server architectures, and integration with your control plane. This approach is ideal for privacy-sensitive applications, offline capabilities, and cost optimization at scale.}

\abstract{This optional chapter explores browser-based inference as a way to achieve zero marginal cost at scale. We cover WebGPU and WebGL technologies, browser inference frameworks (WebLLM, Transformers.js), hybrid client-server architectures, and integration with your control plane. This approach is ideal for privacy-sensitive applications, offline capabilities, and cost optimization at scale.}

% =============================================================================
\section{Why Browser Inference?}
\label{sec:why-browser-inference}
% =============================================================================

% TODO: Motivation for browser-based inference

\subsection{The Traditional Model's Limitations}
\label{subsec:traditional-limitations}

% TODO: Problems with server-only
% - Every request costs GPU time
% - Network latency
% - Privacy concerns
% - Scaling costs

\subsection{The Browser Alternative}
\label{subsec:browser-alternative}

% TODO: Benefits
% - Zero marginal cost
% - Ultra-low latency
% - Perfect privacy
% - Infinite scale (user's device)
% - Offline capable

\subsection{When Browser Inference Makes Sense}
\label{subsec:browser-use-cases}

% TODO: Ideal use cases
% - Privacy-critical applications
% - High-volume, simple tasks
% - Offline-first applications
% - Latency-sensitive features
% - Cost optimization (freemium products)

\begin{svgraybox}
\textbf{Ideal Browser Inference Use Cases:}
\begin{itemize}
\item Medical symptom checkers (privacy)
\item Text classification and sentiment analysis (high volume)
\item Mobile apps in low-connectivity areas (offline)
\item Real-time autocomplete (latency)
\item Freemium products with free tier (cost)
\end{itemize}

\textbf{When to Stick with Server:}
\begin{itemize}
\item Complex reasoning required (>7B model needed)
\item Consistent quality critical
\item Enterprise SLAs required
\item Model must remain proprietary
\end{itemize}
\end{svgraybox}

% =============================================================================
\section{Browser Inference Technologies}
\label{sec:browser-technologies}
% =============================================================================

% TODO: Web standards for AI

\subsection{WebGPU}
\label{subsec:webgpu}

% TODO: Modern GPU API
% - What it is
% - Capabilities
% - Browser support
% - Detection code example

\subsection{WebGL 2.0}
\label{subsec:webgl}

% TODO: Fallback option
% - Performance vs WebGPU
% - When to use

\subsection{WebAssembly (WASM)}
\label{subsec:wasm}

% TODO: CPU inference
% - When GPU unavailable
% - Performance characteristics

\subsection{WebNN (Future)}
\label{subsec:webnn}

% TODO: Upcoming standard
% - Current status
% - Future potential

% =============================================================================
\section{Browser Inference Frameworks}
\label{sec:browser-frameworks}
% =============================================================================

% TODO: Framework comparison

\subsection{WebLLM (MLC-LLM)}
\label{subsec:webllm}

% TODO: Best for production LLMs
% - Features
% - Performance
% - Code example

\begin{programcode}{WebLLM Example}
\begin{lstlisting}[language=JavaScript]
import { CreateMLCEngine } from "@mlc-ai/web-llm";

const engine = await CreateMLCEngine(
    "Phi-3-mini-4k-instruct-q4f16_1",
    {
        initProgressCallback: (progress) => {
            console.log(`Loading: ${progress.progress}%`);
        }
    }
);

// Chat-style API
const reply = await engine.chat.completions.create({
    messages: [
        { role: "user", content: "Write a Python function" }
    ],
    stream: true,
});

// Handle streaming
for await (const chunk of reply) {
    console.log(chunk.choices[0]?.delta?.content || "");
}
\end{lstlisting}
\end{programcode}

\subsection{Transformers.js}
\label{subsec:transformersjs}

% TODO: HuggingFace's solution
% - Ease of use
% - Model support
% - Code example

\subsection{ONNX Runtime Web}
\label{subsec:onnx-web}

% TODO: Cross-platform
% - Custom model support
% - Lower-level control

\subsection{Framework Comparison}
\label{subsec:framework-comparison}

% TODO: Comparison table
% | Feature | Transformers.js | WebLLM | ONNX Runtime |
% | Ease of use | 5/5 | 4/5 | 3/5 |
% | Performance | 3/5 | 5/5 | 4/5 |
% | Streaming | 2/5 | 5/5 | 2/5 |
% | etc. |

% =============================================================================
\section{Model Selection for Browser}
\label{sec:browser-model-selection}
% =============================================================================

% TODO: Size constraints

\subsection{Device Constraints}
\label{subsec:device-constraints}

% TODO: What fits where
% - Mobile: 500MB-1GB models
% - Desktop: 1-4GB models
% - High-end: Up to 7B Q4

\subsection{Recommended Models}
\label{subsec:browser-recommended-models}

% TODO: Specific recommendations
% - TinyLlama for mobile
% - Phi-3-mini for desktop
% - Qwen Coder for code tasks

% =============================================================================
\section{Capability Detection}
\label{sec:capability-detection}
% =============================================================================

% TODO: Detecting what the device can run

\begin{programcode}{Capability Detection}
\begin{lstlisting}[language=JavaScript]
async function detectCapabilities() {
    const capabilities = {
        hasWebGPU: false,
        hasWebGL2: false,
        gpuTier: "none",
        estimatedRAMGB: 4,
        isMobile: false
    };

    // Check WebGPU
    if (navigator.gpu) {
        try {
            const adapter = await navigator.gpu.requestAdapter();
            capabilities.hasWebGPU = !!adapter;
        } catch {}
    }

    // Check WebGL2
    const canvas = document.createElement('canvas');
    capabilities.hasWebGL2 = !!canvas.getContext('webgl2');

    // Detect device type
    capabilities.isMobile = /Android|iPhone|iPad/i.test(navigator.userAgent);

    // Estimate RAM
    if (navigator.deviceMemory) {
        capabilities.estimatedRAMGB = navigator.deviceMemory * 0.5;
    }

    return capabilities;
}
\end{lstlisting}
\end{programcode}

% =============================================================================
\section{Building the Hybrid Client}
\label{sec:hybrid-client}
% =============================================================================

% TODO: Client that chooses browser vs server

\subsection{Architecture}
\label{subsec:hybrid-architecture}

% TODO: Decision flow
% 1. Detect capabilities
% 2. Ask server for routing decision
% 3. Execute in browser or server
% 4. Report analytics

\subsection{Routing Decision API}
\label{subsec:routing-decision}

% TODO: Server-side routing logic

\subsection{Client Implementation}
\label{subsec:client-implementation}

% TODO: Full hybrid client code

% =============================================================================
\section{Control Plane Integration}
\label{sec:browser-control-plane}
% =============================================================================

% TODO: Adding browser support to control plane

\subsection{Browser Routing Endpoint}
\label{subsec:browser-routing-endpoint}

\begin{programcode}{Browser Router (Go)}
\begin{lstlisting}[language=Go]
// internal/browser/router.go

type HybridRouter struct {
    modelRegistry BrowserModelRegistry
    config        HybridConfig
}

type RoutingDecision struct {
    UseBrowser bool              `json:"useBrowser"`
    Reason     string            `json:"reason"`
    ModelInfo  *BrowserModelInfo `json:"modelInfo,omitempty"`
}

func (r *HybridRouter) Route(
    ctx context.Context,
    req RoutingRequest,
) (RoutingDecision, error) {
    // Check if browser inference is enabled
    if !r.config.EnableBrowserInference {
        return RoutingDecision{
            UseBrowser: false,
            Reason:     "browser_inference_disabled",
        }, nil
    }

    // Check device capabilities
    if !r.hasMinimumCapabilities(req.Capabilities) {
        return RoutingDecision{
            UseBrowser: false,
            Reason:     "insufficient_device_capabilities",
        }, nil
    }

    // Check request complexity
    if r.estimateComplexity(req) > ComplexityMedium {
        return RoutingDecision{
            UseBrowser: false,
            Reason:     "request_too_complex",
        }, nil
    }

    // Find compatible browser model
    models, _ := r.modelRegistry.ListCompatible(req.Capabilities)
    if len(models) == 0 {
        return RoutingDecision{
            UseBrowser: false,
            Reason:     "no_compatible_browser_model",
        }, nil
    }

    return RoutingDecision{
        UseBrowser: true,
        Reason:     "optimal_for_browser",
        ModelInfo:  &models[0],
    }, nil
}
\end{lstlisting}
\end{programcode}

\subsection{Browser Model Registry}
\label{subsec:browser-model-registry}

% TODO: Managing browser-capable models

\subsection{Analytics Collection}
\label{subsec:browser-analytics}

% TODO: Tracking browser inference usage

% =============================================================================
\section{Performance and Economics}
\label{sec:browser-economics}
% =============================================================================

% TODO: Expected performance and cost savings

\subsection{Performance Expectations}
\label{subsec:performance-expectations}

% TODO: Tokens/sec by device
% - iPhone: 10-15 tok/s (1B model)
% - M1 MacBook: 25-30 tok/s (3B model)
% - Gaming PC: 35-45 tok/s (7B model)

\subsection{Cost Analysis}
\label{subsec:browser-cost-analysis}

% TODO: Compare server vs hybrid costs
% - 10,000 users example
% - 80% browser, 20% server
% - 78% cost reduction

% =============================================================================
\section{Best Practices}
\label{sec:browser-best-practices}
% =============================================================================

% TODO: Patterns for success

\subsection{Progressive Enhancement}
\label{subsec:progressive-enhancement}

% TODO: Start server, upgrade to browser

\subsection{User Control}
\label{subsec:user-control}

% TODO: Let users choose

\subsection{Battery Awareness}
\label{subsec:battery-awareness}

% TODO: Respect battery status

\subsection{Caching Strategy}
\label{subsec:browser-caching}

% TODO: Model caching

% =============================================================================
\section{Limitations and Challenges}
\label{sec:browser-limitations}
% =============================================================================

% TODO: Honest assessment
% - Model size limits
% - Performance variability
% - Browser compatibility
% - Battery drain
% - Storage quotas

% =============================================================================
\section{Summary}
\label{sec:ch055-summary}
% =============================================================================

\begin{important}{Key Takeaways}
\begin{itemize}
\item Browser inference enables zero marginal cost at scale
\item WebGPU + WebLLM provide production-ready browser inference
\item Hybrid architecture uses browser when possible, server when needed
\item Cost savings of 60-80\% are achievable with hybrid approach
\item Privacy is a killer feature---data never leaves the device
\item UX matters: manage load times and set expectations
\end{itemize}
\end{important}

% =============================================================================
\section*{Problems}
\addcontentsline{toc}{section}{Problems}
% =============================================================================

\begin{prob}
\label{prob:ch055-hybrid-client}
\textbf{Hybrid Client Implementation}\\
Build a complete hybrid inference client in TypeScript that detects device capabilities, consults the control plane for routing decisions, and executes inference either locally or via API.
\end{prob}

\begin{prob}
\label{prob:ch055-cost-comparison}
\textbf{Cost Comparison Analysis}\\
For a hypothetical application with 50,000 users making 20 requests/day, calculate the monthly cost for: (a) server-only, (b) hybrid 50/50, and (c) hybrid 80/20 browser/server. Include hardware, power, and bandwidth costs.
\end{prob}

\begin{prob}
\label{prob:ch055-browser-benchmark}
\textbf{Browser Performance Benchmark}\\
Using WebLLM, benchmark inference speed across three different devices (mobile, laptop, desktop). Document tokens/second, time-to-first-token, and memory usage for each.
\end{prob}

%%%%%%%%%%%%%%%%%%%%%%%% references02.tex %%%%%%%%%%%%%%%%%%%%%%%%%%%%%%
% References for Chapter 2
%%%%%%%%%%%%%%%%%%%%%%%% Springer-Verlag %%%%%%%%%%%%%%%%%%%%%%%%%%

\begin{thebibliography}{99.}

\bibitem{nvidia2024h100} NVIDIA: H100 Tensor Core GPU Datasheet. \url{https://www.nvidia.com/en-us/data-center/h100/} (2024)

\bibitem{nvidia2024a100} NVIDIA: A100 Tensor Core GPU Datasheet. \url{https://www.nvidia.com/en-us/data-center/a100/} (2024)

\bibitem{amd2024mi300} AMD: Instinct MI300X Accelerator. \url{https://www.amd.com/en/products/accelerators/instinct/mi300/mi300x.html} (2024)

\bibitem{apple2024m4} Apple: M4 Chip Architecture. \url{https://www.apple.com/newsroom/2024/05/apple-introduces-m4-chip/} (2024)

\bibitem{runpod2025pricing} RunPod: GPU Cloud Pricing. \url{https://www.runpod.io/gpu-instance/pricing} (accessed December 2025)

\bibitem{vastai2025} Vast.ai: GPU Marketplace. \url{https://vast.ai/} (accessed December 2025)

\bibitem{lambdalabs2025} Lambda Labs: GPU Cloud. \url{https://lambdalabs.com/service/gpu-cloud} (accessed December 2025)

\end{thebibliography}

