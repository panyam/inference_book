% Diagram: Model File Structure
% Used in: Chapter 4, Section 4.1.1 (What a Model File Contains)
\begin{figure}[htbp]
\centering
\begin{tikzpicture}[
    node distance=0.3cm,
    block/.style={rectangle, draw, minimum width=6cm, minimum height=0.7cm, font=\small},
    header/.style={block, fill=blue!20},
    meta/.style={block, fill=green!20},
    tensor/.style={block, fill=orange!20},
    data/.style={block, fill=gray!10}
]

% File structure
\node[header] (hdr) {Header (magic bytes, version, format info)};
\node[meta, below=of hdr] (idx) {Tensor Index (names, shapes, offsets, dtypes)};
\node[meta, below=of idx] (arch) {Architecture Metadata (layers, heads, dims)};
\node[tensor, below=of arch, minimum height=1.5cm] (t1) {Tensor Data: embed\_tokens [vocab $\times$ dim]};
\node[tensor, below=of t1, minimum height=1cm] (t2) {Tensor Data: layer.0.attention.wq [dim $\times$ dim]};
\node[data, below=of t2] (dots) {...hundreds more tensors...};
\node[tensor, below=of dots, minimum height=1cm] (tn) {Tensor Data: lm\_head [dim $\times$ vocab]};

% Arrows showing offsets
\draw[<->, thick, red] (8, -0.35) -- (8, -4.8) node[midway, right, font=\scriptsize] {Offset-based access};

% Labels
\node[left=0.5cm of hdr, font=\scriptsize, align=right] {Fixed\\size};
\node[left=0.5cm of t1, font=\scriptsize, align=right] {Bulk of\\file size};

\end{tikzpicture}
\caption{Anatomy of a model file: metadata enables direct access to tensor data via offsets.}
\label{fig:model-file-structure}
\end{figure}
