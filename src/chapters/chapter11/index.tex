%%%%%%%%%%%%%%%%%%%%% chapter10.tex %%%%%%%%%%%%%%%%%%%%%%%%%%%%%%%%%
%
% Chapter 10: 30B Model Optimization
%
%%%%%%%%%%%%%%%%%%%%%%%% Springer-Verlag %%%%%%%%%%%%%%%%%%%%%%%%%%

\chapter{30B Model Optimization}
\label{ch:30b-optimization}

\abstract*{Moving from 7B to 30B models introduces new challenges. This chapter covers memory management at scale, KV cache optimization, batching strategies for throughput, vLLM tuning, and multi-model serving. We optimize our deployment to efficiently serve 30B models while maintaining acceptable latency.}

\abstract{Moving from 7B to 30B models introduces new challenges. This chapter covers memory management at scale, KV cache optimization, batching strategies for throughput, vLLM tuning, and multi-model serving. We optimize our deployment to efficiently serve 30B models while maintaining acceptable latency.}

% =============================================================================
\section{The 30B Challenge}
\label{sec:30b-challenge}
% =============================================================================

% TODO: What changes at 30B
% - Memory requirements
% - Throughput considerations
% - Hardware requirements

\subsection{Memory Requirements}
\label{subsec:30b-memory}

% TODO: Calculate memory needs
% - Model weights
% - KV cache
% - Activations
% - Overhead

\begin{svgraybox}
\textbf{30B Model Memory Requirements:}

\begin{itemize}
\item FP16 weights: $30B \times 2$ bytes = 60GB
\item INT8 quantized: $30B \times 1$ byte = 30GB
\item INT4 quantized: $30B \times 0.5$ bytes = 15GB
\item KV cache (4K context): ~2-4GB additional
\item Total with overhead (INT4): ~20-24GB
\end{itemize}

\textbf{Recommended Hardware:} RTX 4090 (24GB) or A100 40GB
\end{svgraybox}

% =============================================================================
\section{KV Cache Management}
\label{sec:kv-cache}
% =============================================================================

% TODO: Deep dive into KV cache

\subsection{Understanding KV Cache}
\label{subsec:kv-cache-understanding}

% TODO: What is KV cache
% - Keys and values from attention
% - Why it grows with context
% - Memory formula

\subsection{KV Cache Optimization Techniques}
\label{subsec:kv-optimization}

% TODO: Optimization strategies
% - Paged attention (vLLM)
% - KV cache compression
% - Sliding window attention

\subsection{Context Length Trade-offs}
\label{subsec:context-tradeoffs}

% TODO: Balancing context and throughput
% - Memory vs context
% - Truncation strategies
% - Multi-turn compression

% =============================================================================
\section{Batching Strategies}
\label{sec:batching}
% =============================================================================

% TODO: Efficient batching

\subsection{Static Batching}
\label{subsec:static-batching}

% TODO: Fixed batch sizes
% - Simple to implement
% - Inefficient for variable lengths

\subsection{Dynamic Batching}
\label{subsec:dynamic-batching}

% TODO: Variable batch sizes
% - Padding considerations
% - Timeout-based batching

\subsection{Continuous Batching}
\label{subsec:continuous-batching}

% TODO: vLLM's approach
% - How it works
% - Throughput benefits
% - Implementation considerations

% =============================================================================
\section{vLLM Optimization}
\label{sec:vllm-optimization}
% =============================================================================

% TODO: vLLM-specific tuning

\subsection{Key Parameters}
\label{subsec:vllm-parameters}

\begin{programcode}{vLLM Configuration}
\begin{lstlisting}[language=bash]
# Optimized vLLM launch for 30B model
python -m vllm.entrypoints.openai.api_server \
    --model meta-llama/Llama-3.1-30B-Instruct \
    --dtype float16 \
    --tensor-parallel-size 1 \
    --max-model-len 4096 \
    --gpu-memory-utilization 0.90 \
    --max-num-batched-tokens 4096 \
    --max-num-seqs 128 \
    --block-size 16 \
    --swap-space 4 \
    --enable-prefix-caching

# Key parameters explained:
# --gpu-memory-utilization: Use 90% of VRAM
# --max-model-len: Maximum context length
# --max-num-seqs: Maximum concurrent sequences
# --block-size: KV cache block size
# --swap-space: CPU swap space in GB
# --enable-prefix-caching: Cache common prefixes
\end{lstlisting}
\end{programcode}

\subsection{Memory Utilization}
\label{subsec:memory-utilization}

% TODO: Balancing memory
% - Model weights
% - KV cache allocation
% - Emergency headroom

\subsection{Prefix Caching}
\label{subsec:prefix-caching}

% TODO: System prompt caching
% - What it caches
% - Memory savings
% - Configuration

% =============================================================================
\section{Multi-Model Serving}
\label{sec:multi-model}
% =============================================================================

% TODO: Serving multiple models

\subsection{Model Loading Strategies}
\label{subsec:model-loading}

% TODO: How to manage multiple models
% - Always loaded
% - On-demand loading
% - LRU eviction

\subsection{Memory Sharing}
\label{subsec:memory-sharing}

% TODO: Sharing between models
% - Base model sharing (LoRA)
% - Dynamic adapter loading

% =============================================================================
\section{Profiling and Debugging}
\label{sec:profiling}
% =============================================================================

% TODO: Finding bottlenecks

\subsection{GPU Profiling}
\label{subsec:gpu-profiling}

% TODO: Tools and techniques
% - nvidia-smi
% - Nsight Systems
% - PyTorch profiler

\subsection{Identifying Bottlenecks}
\label{subsec:bottlenecks}

% TODO: Common bottlenecks
% - Memory bandwidth
% - KV cache
% - Tokenization
% - Network I/O

% =============================================================================
\section{Benchmarking}
\label{sec:benchmarking}
% =============================================================================

% TODO: Measuring performance

\subsection{Key Metrics}
\label{subsec:benchmark-metrics}

% TODO: What to measure
% - Throughput (tokens/sec)
% - Latency (TTFT, TBT)
% - Memory usage
% - Cost per token

\subsection{Benchmark Methodology}
\label{subsec:benchmark-methodology}

% TODO: How to benchmark fairly
% - Consistent prompts
% - Warm-up period
% - Multiple runs
% - Statistical significance

% =============================================================================
\section{Control Plane v0.2}
\label{sec:cpv02}
% =============================================================================

% TODO: Summary of v0.2 features
% - Authentication
% - Rate limiting
% - Caching
% - Queuing
% - Multi-model routing

% =============================================================================
\section{Summary}
\label{sec:ch10-summary}
% =============================================================================

\begin{important}{Key Takeaways}
\begin{itemize}
\item 30B models require careful memory management
\item KV cache grows with context length and batch size
\item Continuous batching dramatically improves throughput
\item vLLM's paged attention enables efficient memory use
\item Profiling is essential for identifying bottlenecks
\item Control Plane v0.2 is now production-ready
\end{itemize}
\end{important}

\begin{svgraybox}
\textbf{Part II Complete!}

You now have a production-ready control plane (v0.2) with:
\begin{itemize}
\item Authentication and API key management
\item Rate limiting with per-tier quotas
\item Response caching (exact and semantic)
\item Priority queue with load shedding
\item Optimized 30B model serving
\end{itemize}

In Part III, we'll scale to multi-tenant with 70B models, adding tenant isolation, billing, and distributed inference.
\end{svgraybox}

% =============================================================================
\section*{Problems}
\addcontentsline{toc}{section}{Problems}
% =============================================================================

\begin{prob}
\label{prob:ch10-memory-calc}
\textbf{Memory Calculation}\\
Calculate the total VRAM required to serve a 30B model with INT4 quantization, 8K context length, and batch size of 16. Include KV cache and overhead.
\end{prob}

\begin{prob}
\label{prob:ch10-benchmark}
\textbf{Performance Benchmark}\\
Benchmark a 30B model with different batch sizes (1, 4, 16, 32). Graph throughput vs latency and find the optimal batch size for your use case.
\end{prob}

\begin{prob}
\label{prob:ch10-multi-model}
\textbf{Multi-Model Router}\\
Implement a router that serves multiple models (7B, 13B, 30B) from a single endpoint. Include automatic model selection based on request complexity and user tier.
\end{prob}

%%%%%%%%%%%%%%%%%%%%%%%% references02.tex %%%%%%%%%%%%%%%%%%%%%%%%%%%%%%
% References for Chapter 2
%%%%%%%%%%%%%%%%%%%%%%%% Springer-Verlag %%%%%%%%%%%%%%%%%%%%%%%%%%

\begin{thebibliography}{99.}

\bibitem{nvidia2024h100} NVIDIA: H100 Tensor Core GPU Datasheet. \url{https://www.nvidia.com/en-us/data-center/h100/} (2024)

\bibitem{nvidia2024a100} NVIDIA: A100 Tensor Core GPU Datasheet. \url{https://www.nvidia.com/en-us/data-center/a100/} (2024)

\bibitem{amd2024mi300} AMD: Instinct MI300X Accelerator. \url{https://www.amd.com/en/products/accelerators/instinct/mi300/mi300x.html} (2024)

\bibitem{apple2024m4} Apple: M4 Chip Architecture. \url{https://www.apple.com/newsroom/2024/05/apple-introduces-m4-chip/} (2024)

\bibitem{runpod2025pricing} RunPod: GPU Cloud Pricing. \url{https://www.runpod.io/gpu-instance/pricing} (accessed December 2025)

\bibitem{vastai2025} Vast.ai: GPU Marketplace. \url{https://vast.ai/} (accessed December 2025)

\bibitem{lambdalabs2025} Lambda Labs: GPU Cloud. \url{https://lambdalabs.com/service/gpu-cloud} (accessed December 2025)

\end{thebibliography}

