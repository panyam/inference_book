%%%%%%%%%%%%%%%%%%%%%%% preface.tex %%%%%%%%%%%%%%%%%%%%%%%%%%%%%%%%%
%
% Preface - Author's introduction to the book
%
%%%%%%%%%%%%%%%%%%%%%%%% Springer %%%%%%%%%%%%%%%%%%%%%%%%%%

\preface

% TODO: Write the preface covering:

\section*{Why This Book?}

% The explosion of large language models has created unprecedented demand for
% inference infrastructure. While cloud APIs offer convenience, they come with
% significant trade-offs: ongoing costs, privacy concerns, latency limitations,
% and vendor lock-in.

% This book takes a different approach. Rather than treating models as black boxes
% accessed through APIs, we'll build the infrastructure to host them ourselves---
% from a single 7B model on consumer hardware to a commercial-grade 400B inference
% lab capable of serving thousands of users.

\section*{Who This Book Is For}

% This book is written for software engineers, infrastructure engineers, and
% technical leaders who want to:
% - Understand how AI inference actually works
% - Deploy models on their own infrastructure
% - Build production-grade inference systems
% - Make informed decisions about hardware investments
% - Create multi-tenant AI platforms

% You should have experience with:
% - Programming (we'll use Go for the control plane, Python for ML tooling)
% - REST APIs and distributed systems
% - Basic containerization (Docker)
% - Command-line tools

% You don't need prior ML or AI experience---that's what this book will teach you.

\section*{What You'll Build}

% Across four parts and 18 chapters, you'll build a single Go-based control plane
% that evolves from serving a 7B model on your laptop to orchestrating a 400B
% inference lab. Nothing gets thrown away---each chapter adds capabilities to what
% you've already built.

% By the end, you'll have:
% - A production-ready inference control plane
% - Multi-tenant authentication and rate limiting
% - Intelligent request routing and load balancing
% - Cost tracking and billing infrastructure
% - Distributed inference across multiple GPUs
% - The knowledge to make informed infrastructure decisions

\section*{How to Read This Book}

% This book is designed to be read sequentially. Each part builds on the previous:
% - Part I (Foundations) can stand alone for learning the basics
% - Part II (Production) requires Part I
% - Part III (Multi-Tenant) requires Parts I-II
% - Part IV (Inference Lab) synthesizes everything

% However, experienced readers may jump to specific chapters for reference.

\section*{Code and Resources}

% All code from this book is available at:
% https://github.com/[repository-url]

% The repository includes:
% - Complete control plane source code
% - Example configurations
% - Deployment templates
% - Cost calculators and spreadsheets

\vspace{\baselineskip}
\begin{flushright}\noindent
Location,\hfill {\it Author Name}\\
Month Year\\
\end{flushright}
